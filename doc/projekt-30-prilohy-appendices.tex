% Tento soubor nahraďte vlastním souborem s přílohami (nadpisy níže jsou pouze pro příklad)

% Pro kompilaci po částech (viz projekt.tex), nutno odkomentovat a upravit
%\documentclass[../projekt.tex]{subfiles}
%\begin{document}

% Umístění obsahu paměťového média do příloh je vhodné konzultovat s vedoucím
% \chapter{Obsah přiloženého paměťového média}

% \begin{itemize}
%     \item \textbf{src} -- zdrojové kódy programu
%     \item \textbf{doc} -- text a zdrojové soubory k textu práce
%     \item \textbf{README} -- základní informace o programu
% \end{itemize}

% \chapter{Manuál}

% Pro spuštění programu je nutné do terminálu vložit následující příkaz:

% \begin{lstlisting}
% python3 src/analyze_face_landmarks.py -r REAL-DATA -g GENERATED-DATA
% \end{lstlisting}

% \bigskip

% \noindent Další informace ke spuštění programu se nacházejí na paměťovém médiu v souboru README.

%\chapter{Konfigurační soubor}

%\chapter{RelaxNG Schéma konfiguračního souboru}

%\chapter{Plakát}

% Pro kompilaci po částech (viz projekt.tex) nutno odkomentovat
%\end{document}
